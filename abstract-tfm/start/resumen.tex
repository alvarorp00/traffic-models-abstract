Este estudio explora las interacciones entre diferentes estilos de conducción y cómo estos afectan la eficiencia y seguridad del tráfico con
diferentes tipos de conductores en situaciones de tráfico variables, haciendo énfasis en la influencia de los distintos estilos de conducción,
en la dinámica del tráfico y la seguridad vial.

Mediante el uso de un modelo de simulación de tráfico simple, se analiza la eficacia y rendimiento de los conductores en diferentes escenarios,
evaluando cómo ciertos estilos de conducción pueden ser más eficientes y seguros que otros, dependiendo de las condiciones de la carretera y el tráfico,
para determinar cómo las interacciones entre diferentes estilos de conducción pueden afectar el flujo vehicular y la seguridad vial.

Los resultados muestran que, si bien los conductores agresivos pueden a priori parecer eficientes individualmente, su presencia
en combinación con conductores de otros estilos puede desencadenar una mayor tasa de accidentes y disminuir la eficiencia general del sistema,
no solo en términos de tiempo individual, sino en la cantidad de vehículos que pueden pasar por un punto en un período de tiempo determinado.
Estos hallazgos ofrecen perspectivas valiosas para tener en cuenta en futuras investigaciones y políticas de tráfico,
y subrayan la importancia de entender la dinámica de diferentes estilos de conducción en la planificación y gestión del tráfico.