En este apartado se discutirán los resultados obtenidos y se compararán con los objetivos planteados en la sección \ref{SEC:OBJETIVOS}.

\paragraph{Revisión de los principales hallazgos}

A lo largo de las simulaciones realizadas, identificamos diversas dinámicas en función de la composición de los conductores y las condiciones de saturación de la vía. Específicamente, pudimos constatar que:

\phantom{    }Los conductores arriesgados por sí solos no suelen ser causantes de accidentes excepto en condiciones de alta saturación.
Sin embargo, cuando se combinan con conductores agresivos, la tasa de accidentes aumenta significativamente.

\phantom{    }Una proporción alta de conductores temerarios puede conducir a que los conductores cautos provoquen accidentes.
A pesar de ello, con una proporción menor de conductores temerarios, la tasa de accidentes sigue siendo baja.

\phantom{    }No se observaron mejoras significativas en términos de tiempo y número de conductores que completan la carretera cuando
se incluyó un grupo de conductores agresivos, independientemente de la presencia de conductores cautos o normales.

\phantom{    }En los grupos de conductores donde los agresivos y temerarios eran predominantes, se observó que pese a que el tiempo promedio
de travesía era menor, por contra, la tasa de accidentes era superior y la cantidad de conductores que terminaban la simulación exitosamente
era menor en comparación con las simulaciones con grupos cautos y normales predominantes.
La posible explicación es que los conductores agresivos y temerarios, al conducir de forma más imprudente,
provocan accidentes y situaciones de riesgo que afectan al global de los conductores, saturando la vía y dejando secciones congestionadas, y por tanto,
a la fluidez del tráfico.

\paragraph{Interpretación y especulación}

Aquí se discutirán las implicaciones de los hallazgos y se propondrán posibles aplicaciones prácticas.

\begin{description}
    \item[Relación entre conductores agresivos o temerarios y el resto:] \phantom{-}
    \begin{description}
        \item[Interpretación:] La combinación de estos dos tipos de conductores con los demás parece generar un entorno más propenso a los accidentes.
        Esto puede deberse a una posible falta de previsibilidad en sus acciones, lo que lleva a reacciones en cadena, bien por apurar distancias de seguridad o velocidades,
        lo que en casos en que los tiempos de reacción son críticos puede provocar accidentes (un conductor de este tipo con tiempo de reacción lento 
        difícilmente podrá evitar un accidente si el vehículo de delante frena de forma repentina).
        \item[Especulación:] Si se educara a los conductores sobre los riesgos de conducir de manera agresiva o arriesgada,
        especialmente cuando se combina con otros estilos de conducción, podría haber una disminución en la tasa de accidentes.
    \end{description}

    \item[Influencia de la saturación del tráfico:] \phantom{-}
    \begin{description}
        \item[Interpretación:] A medida que aumenta la saturación, parece haber un aumento en la ocurrencia de accidentes,
        lo que sugiere que el espacio limitado y las mayores interacciones entre vehículos exacerban los efectos de los estilos de conducción.
        \item[Especulación:] Implementar sistemas inteligentes de gestión del tráfico o diseñar infraestructuras viales más adecuadas podría mitigar los efectos negativos de la saturación.
    \end{description}

    \item[Efectos de la diversidad de conductores:] \phantom{-}
    \begin{description}
        \item[Interpretación:] Aunque uno podría pensar que tener una diversidad de estilos de conducción podría ser beneficioso para dispersar el tráfico y mejorar la fluidez,
        los resultados sugieren que la mezcla incorrecta puede tener el efecto opuesto, por ejemplo al incluir conductores agresivos o temerarios.
        \item[Especulación:] Tal vez haya un equilibrio óptimo o proporción de diferentes estilos de conducción que maximice la eficiencia y seguridad del tráfico.
        Las futuras investigaciones podrían centrarse en encontrar esta proporción.
    \end{description}

    \item[Aplicaciones prácticas:] \phantom{-}
    \begin{description}
        \item[Interpretación:] Estos hallazgos pueden ser esenciales para diseñar campañas de concienciación vial o para la formación de conductores.
        \item[Especulación:] Con el avance de la tecnología de vehículos autónomos, estos datos podrían ser cruciales. Si entendemos cómo diferentes personalidades y
        estilos de conducción afectan el flujo del tráfico, podríamos programar vehículos autónomos para adaptarse a estos estilos o incluso para optimizar el tráfico en general.
    \end{description}
\end{description}

\paragraph{Limitaciones}

En este apartado se discutirán las limitaciones del estudio y se propondrán posibles mejoras.

\begin{description}
    \item[Modelo Simplificado:] \phantom{-}
    \begin{description}
        \item[Descripción:] Los modelos de simulación, aunque intentan capturar la esencia del comportamiento del tráfico, son simplificaciones de la realidad. Pueden no incluir todos los factores y variables que influyen en el comportamiento real de los conductores.
        \item[Impacto:] Existen fenómenos observados en la realidad que no se manifiestan en la simulación, y viceversa.
    \end{description}

    \item[Estilos de Conducción Predeterminados:] \phantom{-}
    \begin{description}
        \item[Descripción:] El estudio clasifica a los conductores en categorías discretas (como "cauto" o "agresivo"), lo que podría no reflejar la variedad y continuidad de estilos de conducción en la realidad.
        \item[Impacto:] La categorización puede conducir a interpretaciones simplificadas y no capturar comportamientos intermedios o combinados.
    \end{description}

    \item[Condiciones Estáticas de la Simulación:] \phantom{-}
    \begin{description}
        \item[Descripción:] Las condiciones de la simulación (como la configuración del tráfico o las condiciones meteorológicas) permanecen constantes, a diferencia de las condiciones variables en carreteras reales.
        \item[Impacto:] Los resultados pueden no reflejar completamente cómo reaccionarían los conductores ante cambios en el entorno.
    \end{description}

    \item[No Consideración de Factores Humanos Complejos:] \phantom{-}
    \begin{description}
        \item[Descripción:] Factores como el cansancio, distracciones, decisiones impulsivas o habilidades de conducción no se incorporan en el modelo. 
        Lizbetin \cite{lizbetin2017} sugiere que estos factores tienen un impacto significativo en el comportamiento del conductor.
        \item[Impacto:] La simulación puede subestimar o sobreestimar ciertos comportamientos o resultados en situaciones reales.
    \end{description}

    \item[Generalización de Resultados:] \phantom{-}
    \begin{description}
        \item[Descripción:] Aunque los resultados de la simulación ofrecen perspectivas valiosas, la generalización de estos hallazgos a todos los contextos y situaciones del tráfico podría ser prematura.
        \item[Impacto:] Las recomendaciones o intervenciones basadas en estos resultados deben aplicarse con precaución en contextos diferentes al de la simulación.
    \end{description}
\end{description}

