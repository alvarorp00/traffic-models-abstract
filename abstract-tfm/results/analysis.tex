Puesto que como se ha mencionado al principio de esta sección, el número de datos y gráficas obtenidas
es elevado, se ha optado por mostrar los resultados más relevantes y significativos. Esto implica
que de cara a mostrar los resultados en la carretera poco saturada y muy saturada, se utilizarán los casos más extremos
disponibles (10 conductores cada 500m; 100 conductores cada 500m). En esta sección, el número de conductores accidentados
se indica de forma absoluta, es decir, el número de conductores accidentados, y no como una proporción del total de conductores.

\begin{figure}[Resultados para conductores Cautos y Temerarios, separados]{FIG:CAT_RECK_COMPARE_1}{Comparación de los resultados de las simulaciones para conductores 100\% Cautos y 100\% Temerarios en distintas condiciones de saturación de vía. La simulación muestra que los conductores cautos no han provocado ningún accidente. En cambio, los conductores temerarios sí, y además un menor número de conductores temerarios ha logrado completar la simulación en comparación a los cautos, si bien en un menor tiempo.
    La barra indica la mediana del tiempo que cada conductor toma para completar el recorrido. El número encima de la barra es la mediana del número de conductores de ese tipo
    que han completado la simulación. La línea roja indica la mediana del número de conductores de ese tipo envueltos en accidentes.}
    \subfigure[SBFIG:CAUTOS_LOW_SATURATION]{100\% Cautos - Vía Poco Saturada}{\image{7cm}{5cm}{pictures/sims/sim_1_0.png}} \hspace{1cm}
    \subfigure[SBFIG:CAUTOS_HIGH_SATURATION]{100\% Cautos - Vía Saturada}{\image{7cm}{5cm}{pictures/sims/sim_1_9.png}} \vspace{1cm}
    
    \subfigure[SBFIG:RECLKESS_LOW_SATURATION]{100\% Temerarios - Vía Poco Saturada}{\image{7cm}{5cm}{pictures/sims/sim_5_0.png}} \hspace{1cm}
    \subfigure[SBFIG:RECLKESS_HIGH_SATURATION]{100\% Temerarios - Vía Saturada}{\image{7cm}{5cm}{pictures/sims/sim_5_9.png}}
\end{figure}

Como se ha mencionado anteriormente, en la figura \ref{FIG:ACCIDENTS} se puede observar un pico en el número de accidentes para
los conductores cautos, y en cambio en la figura \ref{FIG:CAT_RECK_COMPARE_1} se puede observar que los conductores cautos no han provocado ningún accidente.

Esto parece que confirma la hipótesis de que los conductores cautos no provocan accidentes, y que por tanto el pico observado en la figura \ref{FIG:ACCIDENTS}
se ha de deber a la influencia de los conductores agresivos. 

\begin{figure}[Resultados para conductores Cautos y Temerarios, conviviendo 1]{FIG:CAT_RECK_9010}{Resultados de simulación con 90\% cautos y 10\% temerarios en condiciones de poca y mucha saturación.}
    \subfigure[SBFIG:CAT_RECK_9010_1]{90\% Cautos + 10\% Temerarios - Vía Poco Saturada}{\image{7cm}{5cm}{pictures/sims/sim_8_0.png}} \hspace{1cm}
    \subfigure[SBFIG:CAT_RECK_9010_2]{90\% Cautos + 10\% Temerarios - Vía Saturada}{\image{7cm}{5cm}{pictures/sims/sim_8_9.png}} \vspace{1cm}
\end{figure}

\begin{figure}[Resultados para conductores Cautos y Temerarios, conviviendo 2]{FIG:CAT_RECK_5050}{Resultados de simulación con 50\% cautos y 50\% temerarios en condiciones de poca y mucha saturación.}
    \subfigure[SBFIG:CAT_RECK_5050_1]{50\% Cautos + 50\% Temerarios - Vía Poco Saturada}{\image{7cm}{5cm}{pictures/sims/sim_6_0.png}} \hspace{1cm}
    \subfigure[SBFIG:CAT_RECK_5050_2]{50\% Cautos + 50\% Temerarios - Vía Saturada}{\image{7cm}{5cm}{pictures/sims/sim_6_9.png}} \vspace{1cm}
\end{figure}

Las figuras \ref{FIG:CAT_RECK_9010} y \ref{FIG:CAT_RECK_5050} muestran que una alta proporción de conductores temerarios puede causar accidentes entre los conductores cautos.
No obstante, si solo hay un $10\%$ de temerarios, los accidentes son mínimos, incluso en alta saturación.

Sorprendentemente, la presencia de estos conductores temerarios no mejora significativamente el tiempo promedio ni la cantidad
de conductores que completan el trayecto, comparado con un sistema de solo conductores cautos. De hecho, en alta saturación,
se reduce la eficiencia debido a accidentes. Por ello, se deduce que tener conductores agresivos en un sistema dominado por cautos no
es beneficioso: en una distribución $90-10$ las diferencias son mínimas, pero en $50-50$ los accidentes aumentan considerablemente.

Si contrastamos ahora los resultados de la mediana del tiempo que tarda cada conductor cauto en completar el trayecto junto con la mediana
del número de accidentes, podemos observar que en la figura \ref{FIG:CAUTIOUS_COMPARISON_ACCIDENTS} se observan las tres líneas correspondientes a las simulaciones
con $100\%$ conductores cautos, $90\%$ conductores cautos y $50\%$ conductores cautos, mostradas en las figuras \ref{FIG:CAT_RECK_COMPARE_1},
\ref{FIG:CAT_RECK_9010} y \ref{FIG:CAT_RECK_5050} respectivamente. Es interesante ver que, efectivamente, la presencia de conductores
temerarios perjudica en todos los casos el tiempo promedio de travesía (en las figuras anteriores mencionadas solo se mostraban los niveles
de saturación más extremos), y que además, a mayor presencia de conductores temerarios y mayor saturación, mayor es el número de accidentes.

\begin{figure}[Mediana de tiempos y accidentes de conductores Cautos]{FIG:CAUTIOUS_COMPARISON_ACCIDENTS}{
    Gráfico que muestra la mediana de tiempos de conducción y la mediana de accidentes para conductores cautos en diferentes configuraciones con conductores temerarios
    y niveles de saturación. Cada línea en la gráfica representa la mediana del tiempo promedio obtenida de las simulaciones para
    una configuración específica, y cada punto en esa línea representa un nivel diferente de saturación de la vía.
    Encima de cada punto, se muestra el número de accidentes de los conductores cautos para ese nivel y configuración.
}
    \image{14cm}{}{pictures/cautious_comparison_with_accidents.png}
\end{figure}

\begin{figure}[Resultados para conductores Normales y Agresivos, separados]{FIG:NRM_AGG_CMP}{Comparación de los resultados de las simulaciones para conductores 100\% Normales y 100\% Agresivos en distintas condiciones de saturación de vía. La simulación muestra que los conductores normales no han provocado ningún accidente. En cambio, los conductores agresivos sí, y además un menor número de conductores agresivos ha logrado completar la simulación en comparación a los normales, si bien en un menor tiempo.}
    \subfigure[SBFIG:NRM_LOW_SATURATION]{100\% Normales - Vía Poco Saturada}{\image{7cm}{5cm}{pictures/sims/sim_2_0.png}} \hspace{1cm}
    \subfigure[SBFIG:NRM_HIGH_SATURATION]{100\% Normales - Vía Saturada}{\image{7cm}{5cm}{pictures/sims/sim_2_9.png}} \vspace{1cm}
    
    \subfigure[SBFIG:AGGRESSIVE_LOW_SATURATION]{100\% Agresivos - Vía Poco Saturada}{\image{7cm}{5cm}{pictures/sims/sim_4_0.png}} \hspace{1cm}
    \subfigure[SBFIG:AGGRESSIVE_HIGH_SATURATION]{100\% Agresivos - Vía Saturada}{\image{7cm}{5cm}{pictures/sims/sim_4_9.png}}
\end{figure}

\begin{figure}[Resultados para conductores Normales y Agresivos, conviviendo 1]{FIG:NRM_AGG_9010}{
    Resultados de simulación con 90\% normales y 10\% agresivos en condiciones de poca y mucha saturación.}
    \subfigure[SBFIG:NRM_AGG_9010_1]{90\% Normales + 10\% Agresivos - Vía Poco Saturada}{\image{7cm}{5cm}{pictures/sims/sim_9_0.png}} \hspace{1cm}
    \subfigure[SBFIG:NRM_AGG_9010_2]{90\% Normales + 10\% Agresivos - Vía Saturada}{\image{7cm}{5cm}{pictures/sims/sim_9_9.png}} \vspace{1cm}
\end{figure}

\begin{figure}[Resultados para conductores Normal y Agresivos, conviviendo 2]{FIG:NRM_AGG_5050}{
    Resultados de simulación con 50\% normales y 50\% agresivos en condiciones de poca y mucha saturación.}
    \subfigure[SBFIG:NRM_AGG_5050_1]{50\% Normales + 50\% Agresivos - Vía Poco Saturada}{\image{7cm}{5cm}{pictures/sims/sim_7_0.png}} \hspace{1cm}
    \subfigure[SBFIG:NRM_AGG_5050_2]{50\% Normales + 50\% Agresivos - Vía Saturada}{\image{7cm}{5cm}{pictures/sims/sim_7_9.png}} \vspace{1cm}
\end{figure}

\begin{figure}[Mediana de tiempos y accidentes de conductores Normales]{FIG:NORMAL_COMPARISON_ACCIDENTS}{
    Gráfico que muestra la mediana de tiempos de conducción y de accidentes para conductores normales en diferentes configuraciones
    con conductores agresivos y niveles de saturación. Cada línea en la gráfica representa la mediana del tiempo promedio obtenida de las simulaciones para
    una configuración específica, y cada punto en esa línea representa un nivel diferente de saturación de la vía.
    Encima de cada punto, se muestra el número de accidentes de los conductores normales para ese nivel y configuración.
}
    \image{14cm}{}{pictures/normal_comparison_with_accidents.png}
\end{figure}

Las figuras \ref{FIG:NRM_AGG_CMP}, \ref{FIG:NRM_AGG_9010} y \ref{FIG:NRM_AGG_5050} muestran que los conductores normales reaccionan de forma similar
a los cautos ante la presencia de conductores agresivos. Es decir, introducir un pequeño o moderado número de conductores agresivos no mejora los resultados
comparados con un sistema de solo conductores normales.

De nuevo, volvemos a hacer una comparación de los resultados de la mediana del tiempo que tarda cada conductor normal en completar el trayecto junto con la mediana
del número de accidentes. Podemos observar los resultados en la figura \ref{FIG:NORMAL_COMPARISON_ACCIDENTS}, en que se observan las tres líneas correspondientes a las simulaciones
con $100\%$ conductores normales, $90\%$ conductores normales y $50\%$ conductores normales, mostradas en las figuras \ref{FIG:NRM_AGG_CMP},
\ref{FIG:NRM_AGG_9010} y \ref{FIG:NRM_AGG_5050} respectivamente. En este caso existe una diferencia más notable entre los resultados de las simulaciones
si se compara con el caso de los conductores cautos. Una presencia del $10\%$ de conductores agresivos genera un muy ligero crecimiento de los accidentes y apenas
afecta al tiempo promedio de travesía para este tipo de conductores. Sin embargo, si la presencia de conductores agresivos aumenta hasta el $50\%$, el número
de accidentes crece, aunque el tiempo total de travesía se reduce ligeramente.

\begin{figure}[Resultados para conductores Arriesgados]{FIG:RSK_AGG_CMP}{Resultados de simulación con 100\% conductores arriesgados en condiciones de poca y mucha saturación.}
    \subfigure[SBFIG:RSK_LOW_SATURATION]{100\% Arriesgados - Vía Poco Saturada}{\image{7cm}{5cm}{pictures/sims/sim_3_0.png}} \hspace{1cm}
    \subfigure[SBFIG:RSK_HIGH_SATURATION]{100\% Arriesgados - Vía Saturada}{\image{7cm}{5cm}{pictures/sims/sim_3_9.png}} \vspace{1cm}
\end{figure}

\begin{figure}[Resultados para conductores Arriesgados y Agresivos, conviviendo]{FIG:RSK_AGG_5050}{Resultados de simulación con 50\% arriesgados y 50\% agresivos en condiciones de poca y mucha saturación.}
    \subfigure[SBFIG:RSK_AGG_5050_1]{50\% Arriesgados + 50\% Agresivos - Vía Poco Saturada}{\image{7cm}{5cm}{pictures/sims/sim_11_0.png}} \hspace{1cm}
    \subfigure[SBFIG:RSK_AGG_5050_2]{50\% Arriesgados + 50\% Agresivos - Vía Saturada}{\image{7cm}{5cm}{pictures/sims/sim_11_9.png}} \vspace{1cm}
\end{figure}

En las figuras \ref{FIG:RSK_AGG_CMP} y \ref{FIG:RSK_AGG_5050}, notamos que los conductores arriesgados raramente causan accidentes por sí solos, excepto en situaciones de alta saturación.
Pero al mezclarse con conductores agresivos, los accidentes aumentan considerablemente y menos conductores logran completar la simulación.
A pesar de esto, el tiempo promedio no se altera significativamente. Este patrón es similar al observado cuando conductores cautos y normales
interactúan con aquellos más agresivos.

\begin{figure}[Resultados para diferentes conductores, conviviendo]{FIG:MIX_DRVS_RSK}{Resultados de simulación con $5\%$ cautos, $15\%$ normales, $60\%$ arriesgados, $15\%$ agresivos y $5\%$ temerarios.}
    \subfigure[SBFIG:MIX_DRVS_RSK_1]{$5\%$ cautos + $15\%$ normales + $60\%$ arriesgados + $15\%$ agresivos + $5\%$ temerarios - Vía Poco Saturada}{\image{7cm}{5cm}{pictures/sims/sim_18_0.png}} \hspace{1cm}
    \subfigure[SBFIG:MIX_DRVS_RSK_2]{$5\%$ cautos + $15\%$ normales + $60\%$ arriesgados + $15\%$ agresivos + $5\%$ temerarios - Vía Saturada}{\image{7cm}{5cm}{pictures/sims/sim_18_9.png}} \vspace{1cm}
\end{figure}

Este análisis muestra una simulación con una mezcla variada de tipos de conductores (figura \ref{FIG:MIX_DRVS_RSK}).
En este caso predominan los conductores arriesgados pero mantienen una presencia notable tanto conductores normales como agresivos y,
en menor medida, cautos y temerarios.

La combinación de todos estos estilos de conducción presenta resultados interesantes. Con poca saturación
no hay una mejora significativa en el tiempo promedio de travesía comparado con otros grupos menos variados,
y la tasa de accidentes sigue siendo notable.

En condiciones de vía saturada el número de accidentes es elevado y el tiempo de travesía se incrementa,
lo que indica que la presencia de conductores agresivos y temerarios afecta negativamente a la fluidez del tráfico,
especialmente cuando hay un predominio de conductores arriesgados.

\begin{figure}[Mediana de tiempos y accidentes de conductores Arriesgados]{FIG:RISKY_COMPARISON_ACCIDENTS}{
    Gráfico que muestra la mediana de tiempos de conducción y accidentes para conductores arriesgados en diferentes configuraciones
    con otros tipos de conductores y niveles de saturación. Cada línea en la gráfica representa la mediana del tiempo promedio obtenida de las simulaciones para
    una configuración específica, y cada punto en esa línea representa un nivel diferente de saturación de la vía.
    Encima de cada punto, se muestra la mediana de accidentes de los conductores arriesgados para ese nivel y configuración.
    El grupo 5-15-60-15-5 es el \% de cautos, normales, arriesgados, agresivos y temerarios respectivamente.
}
    \image{14cm}{}{pictures/risky_comparison_with_accidents.png}
\end{figure}

Si ahora comparamos en la figura \ref{FIG:RISKY_COMPARISON_ACCIDENTS} los resultados de las simulaciones de los conductores arriesgados
en función de la presencia de otros tipos de conductores (siendo ellos el $100\%$, $50\%$ combinado con agresivos o $60\%$ en el caso mixto de
la figura \ref{FIG:MIX_DRVS_RSK}), podemos observar que la presencia de conductores agresivos reduce la mediana del tiempo que toman
para terminar la travesía a costa de aumentar el número de accidentes.

Por otro lado, cuando son este tipo de conductores los que predominan ($100\%$), el número de accidentes es el menor de los tres casos,
aunque son ligeramente más lentos que en presencia de agresivos, pero más rápidos que en el caso mixto. Y por último, el caso mixto
resulta el más lento para este tipo de conductores, si bien el número de accidentes es menor que con un $50\%$ de agresivos pero mayor
que con un $100\%$ de conductores arriesgados.

Una vez que ha quedado claro cómo se comportan los diferentes tipos de conductores en función de la saturación de la vía y de la presencia
de otros tipos de conductores, se puede proceder a analizar los resultados para otros tipos de combinaciones de conductores.

\begin{figure}[Comparación de tiempos y accidentes por tipo de conductor]{FIG:TYPE_COMPARISON_ACCIDENTS}{
    Gráfico que muestra la media de medianas de tiempos de conducción y accidentes para cada tipo de conductor en una configuración 100\% pura y diferentes niveles de saturación.
    Cada línea en la gráfica representa la mediana del tiempo promedio obtenida de las simulaciones para un tipo de conductor en particular, y cada punto en esa línea representa un nivel diferente de saturación de la vía.
    Encima de cada punto, se muestra la mediana de accidentes para ese nivel y tipo de conductor.
}
    \image{14cm}{}{pictures/type_comparison_with_accidents.png}
\end{figure}

En la figura \ref{FIG:TYPE_COMPARISON_ACCIDENTS} se puede observar el comportamiento de cada tipo de conductor en función de la saturación de la vía
de forma aislada. Se puede observar claramente como cada tipo de conductor aparece ordenado de menor a mayor tiempo de travesía, y de mayor a menor
número de accidentes. Esto es, los conductores cautos son los que más tiempo tardan en completar la travesía, y los que menos accidentes provocan,
mientras que los conductores temerarios son los que menos tiempo tardan en completar la travesía, y los que más accidentes provocan.

Además, añadido al dato que se ha visto anteriormente en esta sección, se puede saber que pese a tomar más tiempo en completar la travesía,
un mayor número de estos será capaz de completarla en el mismo lapso de tiempo que los conductores temerarios, ya que estos últimos provocan
más accidentes y por tanto más congestión. Esto último puede observarse en la figura \ref{FIG:COMPARISON}, en la que se muestra la evolución
de los conductores que terminan la simulación en función de la saturación de la vía y de la presencia de otro grupo más agresivo.

\begin{figure}[Mediana de tiempos y accidentes en diferentes distribuciones]{FIG:CONFIG_COMPARISON_ACCIDENTS}{
    Gráfico que muestra la media de medianas de tiempos de conducción y la mediana de accidentes en diferentes configuraciones de conductores y niveles de saturación.
    Se calcula la mediana para cada valor en función del tipo de conductor y la saturación, y luego se calcula la media de esas medianas.
    Cada línea en la gráfica representa el tiempo promedio obtenido de las simulaciones para una configuración específica, y cada punto en esa línea representa un nivel diferente de saturación de la vía.
    Encima de cada punto, se muestra la media de accidentes para ese nivel y configuración.
    El grupo 40-30-15-10-5 es el \% de cautos, normales, arriesgados, agresivos y temerarios respectivamente. De igual forma para el grupo
    5-10-15-30-40.
}
    \image{14cm}{}{pictures/config_comparison_with_accidents.png}
\end{figure}

En la figura \ref{FIG:CONFIG_COMPARISON_ACCIDENTS} se puede observar una comparativa de los resultados de las simulaciones para tres tipos de situaciones,
una en la que hay la misma presencia de cada tipo de conductor (equiprobable), otra en la que hay un $40\%$ de cautos, $30\%$ de normales,
$15\%$ de arriesgados, $10\%$ de agresivos y $5\%$ de temerarios, y otra en la que se sigue una distribución inversa a esta última.

La simulación con más conductores cautos y normales genera menos accidentes y tiempos similares a la equiprobable,
aunque esta última tiene más accidentes. En contraste, la distribución con predominio de conductores arriesgados y agresivos provoca más accidentes,
pero es la más rápida en conjunto.

Un detalle importante sobre los gráficos que incluyen resultados con conductores temerarios (y agresivos) es que no son tan estables como
cuando se trata de medianas en los otros grupos, principalmente debido a lo que se mostró en la figura \ref{FIG:SUCCESS_RATE}, donde el \%
de simulaciones exitosas ($<10\%$ de accidentes) en las que estos conductores están presentes con gran proporción es menor que en los otros grupos,
por lo que menos simulaciones se tienen en cuenta para calcular la mediana.

% \begin{figure}[Mediana de conductores cautos y temerarios que completaron la simulación]{FIG:CT_FINISHING}{
%     Gráfico que muestra la mediana de conductores cautos y temerarios que lograron completar la simulación en diferentes configuraciones. 
%     Se presenta una comparación directa entre diferentes proporciones de conductores cautos y temerarios, ayudando a entender cómo estos dos tipos de conductores se desempeñan bajo diferentes circunstancias.
% }
%     \image{14cm}{}{pictures/cautious_reckless_finishing_comparison.png}
% \end{figure}

% \begin{figure}[Mediana de conductores normales y agresivos que completaron la simulación]{FIG:NA_FINISHING}{
%     Gráfico que muestra la mediana de conductores normales y agresivos que lograron completar la simulación en diferentes configuraciones. 
%     Se presenta una comparación directa entre diferentes proporciones de conductores normales y agresivos, revelando las diferencias en el desempeño entre estos dos grupos.
% }
%     \image{14cm}{}{pictures/normal_aggressive_finishing_comparison.png}
% \end{figure}

\begin{figure}[Conductores que terminan en Cautos/Temerarios y Normales/Agresivos]{FIG:COMPARISON}{
    Comparativa del número de conductores que terminan exitosamente la simulación para las combinaciones de conductores cautos y temerarios, y normales y agresivos.
    En la primera fila se muestran los resultados para combinaciones de cautos y temerarios, y en la segunda fila para combinaciones de normales y agresivos.
    En la primera columna se muestra la media de las medianas para las distintas condiciones de saturación de la vía, y en la segunda columna se muestra
    la evolución de las medianas para cada condición de saturación. Todo ello distinguiendo los dos tipos de conductores en cada caso.
    }
    % Primera fila: Cautos y Temerarios
    \subfigure[SBFIG:BAR_CAUT_RECK]{Gráfico de barras - Cautos y Temerarios}{\image{7cm}{5cm}{pictures/cautious_reckless_finishing_comparison.png}} \hspace{1cm}
    \subfigure[SBFIG:EVOL_CAUT_RECK]{Evolución de finalizaciones - Cautos y Temerarios}{\image{7cm}{5cm}{pictures/evolution_finishing_comparison_cautious_reckless.png}} \vspace{1cm}

    % Segunda fila: Normales y Agresivos
    \subfigure[SBFIG:BAR_NORM_AGGRESS]{Gráfico de barras - Normales y Agresivos}{\image{7cm}{5cm}{pictures/normal_aggressive_finishing_comparison.png}} \hspace{1cm}
    \subfigure[SBFIG:EVOL_NORM_AGGRESS]{Evolución de finalizaciones - Normales y Agresivos}{\image{7cm}{5cm}{pictures/evolution_finishing_comparison_normal_aggressive.png}}
\end{figure}

Finalmente, en la figura \ref{FIG:COMPARISON} se puede observar la evolución del número de conductores que terminan la simulación
en dos casos distintos: en el primero, se muestran los resultados para combinaciones de cautos y temerarios, y en el segundo para combinaciones de normales y agresivos.

El gráfico de barras muestra la media de las medianas de los resultados de las simulaciones en las distintas condiciones de saturación de la vía,
donde se observa de forma absoluta que cuando hay un $100\%$ de conductores cautos hay un mayor número de conductores que terminan la simulación
que en presencia de un $10\%$ de conductores temerarios (menos incluso que si se hace la proporción del $90\%$ correspondiente al caso con el $100\%$).
Y se ve cómo este número de conductores que terminan la simulación disminuye considerablemente
cuando hay un $50\%$ de conductores temerarios (cuando teóricamente debería ser la mitad del caso con el $100\%$).
Otro hecho interesante es cómo una presencia del $100\%$ de temerarios no obtiene mejores valores absolutos
que un $50\%$ de temerarios (en presencia de cautos), lo que indica que la presencia de conductores cautos ayuda a que los temerarios terminen la simulación.

Respecto al caso de los conductores normales y agresivos, se puede observar una situación similar a diferencia del caso con presencia del $100\%$ de agresivos,
que obtiene mejores valores absolutos que el caso con presencia del $50\%$ de agresivos para este tipo de conductor (siendo esperable pues, en teoría, deberían finalizar el doble
que en el caso con presencia del $50\%$, algo que no ocurre en el caso de los temerarios).

En el segundo tipo de gráfico se muestra la evolución de estas medianas en función de la saturación de la vía. En este caso se puede observar
que se mantiene el resultado global que indica el gráfico de barras.
