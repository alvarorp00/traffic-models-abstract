Tras llevar a cabo el proceso de simulación descrito en la metodología, se han obtenido
una serie de datos que arrojan algo de luz sobre el comportamiento y evolución de los
conductores en diferentes escenarios de tráfico. Recuérdese que se han realizado 18 simulaciones,
cada una con sus propias condiciones y parámetros, permitiendo explorar una variedad de situaciones y obtener una visión completa del sistema.

Una de las primeras impresiones se obtiene al analizar en la figura \ref{FIG:SUCCESS_RATE} la evolución del porcentaje de
simulaciones que lograron terminar sobre el total de simulaciones ejecutadas que se detallaron en la tabla \ref{TB:SIMULATION_STRATEGIES}.
(200 veces se repitió la simulación para cada configuración de tráfico). Esta gráfica brinda una idea de cómo se comporta el sistema
y cómo la densidad de tráfico y la saturación pueden influir en la viabilidad de las simulaciones.
En la figura se puede observar que, en general, las simulaciones con la densidad de tráfico más baja
(10 vehículos en 500 metros) obtienen un porcentaje de éxito que, salvo en los casos de conductores cautos y normales,
decae rápidamente. También se puede observar la pequeña tasa de éxito de las simulaciones donde el grupo predominante
son los conductores más agresivos, lo que sugiere que este tipo de conductores pueden ser más propensos a
provocar accidentes o situaciones de riesgo.

\begin{figure}[Simulaciones exitosas]{FIG:SUCCESS_RATE}{Porcentaje de simulaciones exitosas, i.e. que lograron terminar con un ratio de accidentes < 10\% en
    las distintas configuraciones de saturación, en función de la densidad de tráfico introducida. Cada punto representa la mediana de los resultados obtenidos
    de las 200 simulaciones realizadas para cada configuración, y cada línea cada una de las distintas simulaciones realizadas.}
    \image{15cm}{}{pictures/percentages.png}
\end{figure}

A continuación, se desglosarán los resultados obtenidos para cada métrica, ofreciendo un análisis y comparativas
para los escenarios simulados. Cabe destacar que, debido a la gran cantidad de datos obtenidos, se ha optado por
mostrar los resultados más relevantes y significativos con el fin de facilitar la interpretación de los mismos.