La investigación presentada ha proporcionado valiosas ideas sobre cómo los diferentes estilos de conducción pueden influir en la dinámica del tráfico. Sin embargo, como todo modelo, tiene sus limitaciones inherentes, lo que nos conduce a varias direcciones prometedoras para futuros trabajos.

\subparagraph{Modelo más complejo}

Aunque este estudio abordó varios comportamientos de conducción, es esencial considerar un modelo que
incorpore más variables humanas, como distracciones, fatiga o influencia del clima, para obtener un panorama más completo de las dinámicas reales
del tráfico.

\subparagraph{Entorno más diverso}

El modelo podría extenderse para abordar diferentes tipos de infraestructuras viales,
desde carreteras rurales hasta complejos intercambios urbanos, con el fin de validar si los patrones observados se mantienen consistentes
en diferentes contextos.

\subparagraph{Interacción con vehículos autónomos}

A medida que la tecnología avanza, se espera que los vehículos autónomos se incorporen cada vez
más al tráfico convencional. Investigar cómo estos vehículos interactuarán con conductores humanos, especialmente los de comportamientos extremos,
será crucial.

\subparagraph{Políticas y normativas basadas en datos}

Con una comprensión más profunda de estas interacciones, se puede trabajar en el diseño de políticas
y regulaciones más efectivas que fomenten estilos de conducción seguros y eficientes.

\subparagraph{Herramientas de educación y concienciación}

Basándose en los resultados, se pueden desarrollar programas educativos dirigidos a enseñar a los
conductores sobre los riesgos asociados con ciertos comportamientos, utilizando simulaciones y escenarios derivados directamente del estudio.

En última instancia, la finalidad es contribuir a la creación de sistemas de transporte más seguros, eficientes y resilientes,
adaptados a los desafíos emergentes del siglo XXI. Para ello, la colaboración interdisciplinaria entre ingenieros, psicólogos, urbanistas y otros expertos
será esencial.
