El análisis detallado de las simulaciones de tráfico ha arrojado luz sobre cómo diferentes estilos de conducción interactúan entre sí y cómo estos influyen en la eficiencia y seguridad del tráfico.
Una conclusión clave es que, si bien los conductores agresivos pueden parecer eficientes individualmente, su presencia en combinación con conductores de
otros estilos puede desencadenar una mayor tasa de accidentes y disminuir la eficiencia general del sistema.
Esta dinámica es particularmente pronunciada en situaciones de alta saturación.

Además, es esencial considerar la simplicidad del modelo de simulación. Aunque ha proporcionado resultados valiosos, tiene sus propias limitaciones.
Los factores humanos complejos, las condiciones cambiantes y la variabilidad real de los estilos de conducción no están completamente representados
en el modelo. A pesar de estas limitaciones, los resultados ofrecen una dirección clara para futuras políticas y estrategias a seguir:
equilibrar la eficiencia y la seguridad requiere una comprensión más profunda de cómo interactúan los diferentes estilos de conducción.

En resumen, este estudio subraya la importancia de considerar el comportamiento del conductor en la gestión y planificación del tráfico.
A medida que nuestras ciudades crecen y la infraestructura de transporte se convierte en una parte crucial de la vida urbana, las decisiones basadas
en la comprensión de las interacciones entre diferentes estilos de conducción pueden resultar fundamentales para aliviar los atascos, reducir accidentes
y mejorar la movilidad general. Con la integración de tecnologías avanzadas, como la conducción autónoma y sistemas de monitoreo en tiempo real,
se presenta una oportunidad única de aplicar estas lecciones aprendidas. Sin embargo, el éxito de estas tecnologías y estrategias dependerá de cómo abordemos
y consideremos la diversidad de comportamientos en la carretera. Por lo tanto, entender y abordar estas dinámicas no es solo una recomendación,
sino una necesidad imperativa para las futuras generaciones de sistemas de transporte.