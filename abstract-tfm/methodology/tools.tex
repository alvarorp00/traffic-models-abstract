\paragraph{Lenguaje de Programación}
Se optó por utilizar \textit{Python} como lenguaje principal para desarrollar el modelo de simulación. La elección de \textit{Python} se basa en su versatilidad y facilidad de programación, lo que permitió una rápida iteración y prueba de ideas. Si bien no es el lenguaje más eficiente en términos de tiempo de ejecución, la eficiencia de desarrollo compensó esta limitación.

\paragraph{Bibliotecas Principales}
Para llevar a cabo las simulaciones y análisis, se emplearon las siguientes bibliotecas:
\begin{itemize}
    \item \textit{Numpy}: Utilizada para operaciones matemáticas y manipulación de datos.
    \item \textit{Scipy}: Proporciona herramientas y algoritmos matemáticos para operar con \textit{Python}.
    \item \textit{TensorFlow Probability}: Usada para operaciones estadísticas y probabilísticas avanzadas.
    \item \textit{Matplotlib}: Empleada para visualizar resultados y generar gráficos pertinentes.
    \item \textit{Logging}: Esta biblioteca fue esencial durante la fase de desarrollo, permitiendo un registro detallado y seguimiento de los eventos durante las simulaciones, facilitando la identificación y corrección de errores.
    \item \textit{Pandas}: Utilizada para el análisis de datos y la generación de estadísticas.
\end{itemize}

\paragraph{Control de Versiones}
Se empleó \textit{Git} como herramienta de control de versiones, garantizando la trazabilidad y el registro de cambios en el código. Esta herramienta resultó crucial para poder retornar a versiones estables del código ante posibles fallos o problemas, asegurando así la integridad y continuidad del proyecto.

\paragraph{Hardware y Plataforma de Simulación}
Las simulaciones se llevaron a cabo en un servidor dedicado con las siguientes especificaciones:
\begin{itemize}
    \item \textbf{RAM}: 32 GB a 2400MHz.
    \item \textbf{CPU}: AMD Ryzen 7 3800X 8-core a 3.9 GHz.
    \item \textbf{Disco Primario}: SSD M.2 1TB con velocidades de 1800MB/s tanto en lectura como en escritura.
\end{itemize}
Aunque se consideró la opción de usar \textit{Google Colab} para realizar las simulaciones, se descartó debido a las restricciones de tiempo de ejecución. La versión gratuita de \textit{Colab} limita las sesiones a 12 horas, y aunque la versión \textit{Pro} extiende este límite a 24 horas, ejecutar las simulaciones en un servidor propio ofreció una mayor flexibilidad y control, permitiendo además ejecutar múltiples simulaciones en paralelo.
