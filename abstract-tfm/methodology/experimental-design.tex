\paragraph{Definición del Experimento}
El propósito central de este experimento es analizar la influencia de la agresividad en la conducción sobre el tiempo global medio de viaje.
Específicamente, se busca comprobar si un aumento en la proporción de conductores agresivos disminuye o aumenta este tiempo.

Asímismo, se pretende explorar si estos conductores se benefician únicamente cuando hay una mayor presencia de conductores cautos.
Aunque se espera que los conductores agresivos tengan tiempos de viaje global más cortos,
se postula que este tipo de estrategia más agresiva seguida por estos conductores podría tener un efecto negativo no tanto en el tiempo de viaje,
que debería ser menor, sino en cuántos conductores logran completar el viaje durante el tiempo de simulación en comparación con otros conductores
que siguen estrategias más cautelosas.

Es decir, se espera que los atascos producidos tanto por los accidentes como por la congestión del tráfico
sean más frecuentes en las simulaciones con una mayor proporción de conductores agresivos.

\paragraph{Parámetros Fijos}
\begin{itemize}
    \item Vía: longitud de 5000 metros, sección de 500 metros, 2 carriles.
    \item Instantes de Simulación: 1000.
    \item Proporción de tiempos de reacción, tipos de coche, distancias de seguridad: descrito detalladamente en la sección del modelo de simulación.
    \item Presencia de conductores por carril (al ser introducidos a la vía): $15\%$ izquierdo, $85\%$ derecho.
\end{itemize}

\paragraph{Parámetros Variables}
\begin{itemize}
    \item Número de conductores en la primera sección: se incrementará de 10 en 10, desde 10 hasta 100.
    \item Proporción de tipos de conductores: Se probarán distintas combinaciones y gradientes, detallado en la tabla \ref{TB:SIMULATION_STRATEGIES}.
\end{itemize}

\paragraph{Configuraciones de las Simulaciones}

Se realizarán 18 simulaciones, con distintas configuraciones de tipos de conductores, que se muestran en la tabla \ref{TB:SIMULATION_STRATEGIES}.

\begin{table}[Simulaciones de Estrategias]{TB:SIMULATION_STRATEGIES}{Composición de las estrategias de simulación.}
    \begin{tabular}{| c | c | c | c | c | c |}
        \hline
        \textbf{Simulación} & \textbf{Cautos} & \textbf{Normal} & \textbf{Arriesgado} & \textbf{Agresivo} & \textbf{Temerario} \\
        \hline
        1 & 100\% & 0\% & 0\% & 0\% & 0\% \\
        \hline
        2 & 0\% & 100\% & 0\% & 0\% & 0\% \\
        \hline
        3 & 0\% & 0\% & 100\% & 0\% & 0\% \\
        \hline
        4 & 0\% & 0\% & 0\% & 100\% & 0\% \\
        \hline
        5 & 0\% & 0\% & 0\% & 0\% & 100\% \\
        \hline
        6 & 50\% & 0\% & 0\% & 0\% & 50\% \\
        \hline
        7 & 0\% & 50\% & 0\% & 50\% & 0\% \\
        \hline
        8 & 90\% & 0\% & 0\% & 0\% & 10\% \\
        \hline
        9 & 0\% & 90\% & 0\% & 10\% & 0\% \\
        \hline
        10 & 50\% & 50\% & 0\% & 0\% & 0\% \\
        \hline
        11 & 0\% & 0\% & 50\% & 50\% & 0\% \\
        \hline
        12 & 20\% & 20\% & 20\% & 20\% & 20\% \\
        \hline
        13 & 40\% & 30\% & 15\% & 10\% & 5\% \\
        \hline
        14 & 5\% & 10\% & 15\% & 30\% & 40\% \\
        \hline
        15 & 25\% & 25\% & 25\% & 25\% & 0\% \\
        \hline
        16 & 0\% & 25\% & 25\% & 25\% & 25\% \\
        \hline
        17 & 0\% & 33\% & 34\% & 33\% & 0\% \\
        \hline
        18 & 5\% & 15\% & 60\% & 15\% & 5\% \\
        \hline
    \end{tabular}
\end{table}

\paragraph{Métricas de Evaluación}
\begin{itemize}
    \item Tiempo total de viaje: evaluará la mediana del tiempo total que toma a todos los vehículos recorrer la distancia completa.
    \item Número de accidentes: registra la mediana del total de accidentes ocurridos durante la simulación.
    \item Número de vehículos que completaron el viaje: evaluará la mediana del número de vehículos que lograron completar el viaje.
\end{itemize}

\paragraph{Procedimiento}

\subparagraph{Inicialización}

Estado inicial de la carretera: vacía.

El tiempo evolucionará de forma discreta. Es decir, en el modelo las variables cambiarán en instantes específicos y separados,
en lugar de cambiar continuamente: en lugar de tener un flujo constante de tiempo, el tiempo se divide en unidades o pasos separados.

En el contexto de simulaciones, trabajar con tiempo discreto permite analizar el comportamiento del sistema en momentos determinados,
facilitando el procesamiento y la comprensión de los eventos. Cada instante se corresponderá con un segundo de tiempo real.

En cada segundo (cada instante en las simulaciones), se evalúa si el número de conductores en la primera sección es inferior al límite máximo establecido para el caso en consideración.
Si la condición se cumple, se introduce un nuevo conductor a la vía.

\subparagraph{Ejecución}

Dado que el tiempo es discreto, en cada iteración se considera un \textit{instante congelado} del tiempo.
La \textit{velocidad de reacción} del conductor determina qué instante discreto (en segundos) del pasado de la simulación se escogerá
para tomar decisiones (1-5 instantes anteriores, como se explicó en la sección \ref{SEC:SIMULATIONMODELD}).
Para actualizar la posición y velocidad de cada conductor, se invoca al algoritmo \ref{ALG:SPEED_LANE_UPDATE}.

\subparagraph{Métodos de Actualización}

Las decisiones específicas sobre la actualización de la velocidad y el carril se basan en los algoritmos \ref{ALG:SPEED_LANE_UPDATE}, \ref{ALG:SPEED_UPDATE} y \ref{ALG:LANE_UPDATE}.
Para una visión más detallada del código y la implementación completa, se puede acceder al repositorio del proyecto en GitHub\footnote{Enlace del repositorio: \url{https://github.com/alvarorp00/traffic-models}}.

\begin{itemize}
    \item Si no hay conductor delante, el conductor intentará alcanzar su velocidad máxima.
    \item Si hay un conductor delante:
    \begin{itemize}
        \item Evalúa si puede adelantar, es decir, si su velocidad es mayor y el carril contiguo está libre.
        \item Si no puede adelantar, ajusta su velocidad para mantener una distancia segura con el conductor de delante.
    \end{itemize}
\end{itemize}

\begin{itemize}
    \item Determina si el conductor puede o debe cambiar de carril basándose en condiciones como la presencia de otro conductor delante,
    la existencia de espacio libre en el carril contiguo, el espacio libre en el carril actual y el tiempo que lleva en el carril actual
    (hay un mínimo de 8-12 segundos elegidos uniformemente antes de que el conductor pueda cambiar de carril para evitar cambios de carril constantes).
    \item Si no hay necesidad o no es posible cambiar de carril, el conductor permanecerá en su carril actual.
\end{itemize}


\begin{algorithmN}[ALG:SPEED_LANE_UPDATE]{Actualización de velocidad y carril.}{Método para actualizar la velocidad y el carril del conductor.}
    \SetKwInOut{Input}{Entrada}
    \SetKwInOut{Output}{Salida}
    \Input{$driver$, conductor actual}
    \Input{$state$, estado del modelo de tráfico}
    \Output{$driver$, conductor actualizado}
    % $\text{Obtiene conductor delante y detrás de driver}$\;
    $dfront \leftarrow \text{driver\_at\_front(state)}$\;
    $dback \leftarrow \text{driver\_at\_back(state)}$\;
    $lane \leftarrow \textbf{lane\_update(driver, state)}$\;

    \If{$lane \ne driver.lane$} {
        $dfront \leftarrow \text{driver\_at\_front(state)}$\;
        $dback \leftarrow \text{driver\_at\_back(state)}$\;
    }

    $speed \leftarrow \textbf{speed\_update(driver, dfront, dback)}$\;
    $driver.speed \leftarrow speed$\;
    \Return{$driver$}\;
\end{algorithmN}

\begin{algorithmN}[ALG:SPEED_UPDATE]{Actualización de la velocidad.}{Método para actualizar la velocidad del conductor.}
    \SetKwInOut{Input}{Entrada}
    \SetKwInOut{Output}{Salida}
    \Input{$driver$, conductor actual}
    \Input{$dfront$, conductor al frente}
    \Input{$dback$, conductor atrás}
    \Output{$speed$, Velocidad actualizada}

    \uIf{$\exists \, dfront$}{
        $safe\_distance \leftarrow \text{calculate\_safe\_distance(driver)}$\;
        $distance\_to\_front \leftarrow \text{calculate\_distance\_to\_front(driver, dfront)}$\;
        $desired\_speed \leftarrow \text{calculate\_desired\_speed(driver, safe\_distance, distance\_to\_front)}$\;    
    }\Else{
        $desired\_speed \leftarrow \text{get\_max\_speed(driver)}$\;
    }
    \Return{$\min(desired\_speed, \text{get\_max\_speed\_by\_type(driver)})$}\;
\end{algorithmN}
    
\begin{algorithmN}[ALG:LANE_UPDATE]{Actualización del carril.}{Método para actualizar el carril del conductor.}
    \SetKwInOut{Input}{Entrada}
    \SetKwInOut{Output}{Salida}
    \Input{$driver$, conductor actual}
    \Input{$state$, estado de la simulación}
    \Output{$lane$, carril actualizado}

    $current\_lane \leftarrow driver.\text{get\_current\_lane()}$\;
    \uIf{$\text{can\_overtake(driver, state)}$}{
        $next\_lane \leftarrow \text{next\_lane(current\_lane)}$\;
        $can\_switch\_lane \leftarrow \text{can\_switch\_to\_lane(driver, state, next\_lane)}$\;
        \uIf{$can\_switch\_lane$ and $driver.time\_in\_lane == 0$}{
            $driver.time\_in\_lane \leftarrow \text{calculate\_time\_in\_lane(driver)}$\;
            $lane \leftarrow next\_lane$\;
        }
        \Else{
            $lane \leftarrow current\_lane$\;
        }
    }
    \uElseIf{$\text{can\_return\_lane(driver, state, current\_lane)}$}{
        $lane \leftarrow \text{prev\_lane(current\_lane)}$\;
    }
    \Else{
        $lane \leftarrow current\_lane$\;    
    }
    \Return{$lane$}\;
\end{algorithmN}


\paragraph{Finalización y Recopilación de Datos}

Al concluir cada simulación, se procede a una fase de recopilación y almacenamiento de datos críticos para el análisis posterior.
Los pasos que se llevan a cabo se detallan a continuación.

\subparagraph{Finalización de la Simulación} 

Una vez que se alcanza el término de la simulación, ya sea por tiempo o por cumplimiento de alguna condición preestablecida, se detiene la ejecución y se pasa al proceso de consolidación de los datos obtenidos.

\subparagraph{Procesamiento de los Resultados}

Antes de guardar los datos, se realizan cálculos intermedios para obtener métricas relevantes. En este caso, dada la naturaleza de los datos y para reducir el impacto de valores atípicos o extremos, se decide usar la mediana. Por lo tanto, se calculan las medianas de:

\begin{itemize}
    \item Tiempo que tardan los conductores en completar el trayecto.
    \item Accidentes ocurridos durante la simulación.
    \item Número de conductores que han logrado finalizar el recorrido.
\end{itemize}

Adicionalmente, estos cálculos se segmentan y se realizan por cada tipo de conductor, permitiendo un análisis más detallado y categorizado.

\subparagraph{Almacenamiento de Datos}

Los resultados de cada simulación se almacenan en un fichero CSV, permitiendo preservar y analizar posteriormente los datos obtenidos. Además, se generan gráficas para visualizar los resultados de forma más intuitiva.
