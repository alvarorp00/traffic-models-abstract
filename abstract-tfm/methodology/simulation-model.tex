\paragraph{Premisas Generales}

% Breve introducción a la necesidad y el objetivo del modelo

Puesto que el objetivo del modelo es analizar el comportamiento de los conductores en un entorno controlado, es necesario definir
un conjunto de premisas generales que permitan la simulación de escenarios realistas -si bien no excesivamente complejos, puesto que con un modelo
simple es posible obtener resultados valiosos- de los cuales se puedan extraer conclusiones relevantes.

% La importancia de la representatividad y precisión en la simulación del tráfico vehicular

La representatividad y precisión de los datos obtenidos de las simulaciones es fundamental para poder realizar un análisis detallado y
extraer conclusiones válidas. Por ello, es necesario que el modelo sea capaz de simular escenarios que reflejen fielmente las condiciones
reales de las carreteras, incluyendo factores como la densidad vehicular, la velocidad, la distancia de seguridad, la agresividad de los conductores,
diferentes tipos de vehículos y los tiempos de reacción, obviando factores como el clima o el estado de la carretera, que pueden influir en el
comportamiento de los conductores pero que no son relevantes para el propósito del modelo, además de ser más difíciles de simular.

\paragraph{Tipos de Conductores}

% Cautos, normales, arriesgados, agresivos, temerarios

Como se mencionó en la sección \ref{SEC:BACKGROUND}, existen diferentes tipos de conductores, que varían desde los más cautos hasta los más agresivos.
Para poder realizar un análisis detallado, es necesario definir un conjunto de tipos de conductores que permitan representar la diversidad de
comportamientos que se pueden observar en las carreteras. En este caso, se han definido cinco tipos de conductores, que se describen en la tabla
\ref{TB:DRIVER_TYPES}.

\begin{table}[Tipos de conductores]{TB:DRIVER_TYPES}{Características de los conductores del modelo.}
    \begin{tabular}{| c | c | c |}
        % {\textwidth}{@{\extracolsep{\fill}} | c | c | c | c | }
        \hline
        \textbf{Tipo} & \textbf{Distancia de Seguridad} & \textbf{Distancia de Visión}\\
        \hline
        \textbf{Cauto} & 37.5 metros & 100 metros \\
        \hline
        \textbf{Normal} & 31.25 metros & 100 metros \\
        \hline
        \textbf{Arriesgado} & 25 metros & 100 metros \\
        \hline
        \textbf{Agresivo} & 18.75 metros & 100 metros \\
        \hline
        \textbf{Muy agresivo} & 12.5 metros & 100 metros \\
        \hline
    \end{tabular}
\end{table}

Respecto a la velocidad máxima establecida, que dependerá de cada vehículo y que se puede consultar en la tabla \ref{TB:VEHIC_TYPES}, se han considerado los
valores máximos a los que podrán circular cada conductor (que dependerá del tipo de vehículo que conduzca), y que se muestran en la tabla \ref{TB:DRIVER_TYPES}.
Además, se ha establecido un campo de visión para la toma de decisiones de 100 metros, igual para todos los conductores.

Por último, es necesario definir el tiempo de reacción de cada tipo de conductor, para poder simular el tiempo que tarda un conductor en reaccionar ante un cambio,
estableciéndose cuatro tipos de reacciones, independientes del tipo de conductor, que se muestran en la tabla \ref{TB:REACTION_TYPES} y donde se puede observar
la proporción de conductores que tienen cada tipo de reacción.

\begin{table}[Tipos de tiempos de reacción]{TB:REACTION_TYPES}{Características de los tiempos de reacción del modelo. Se muestran los tiempos de reacción de cada tipo, y la proporción de conductores que los tienen.}
    \begin{tabular}{| c | c | c |}
        % {\textwidth}{@{\extracolsep{\fill}} | c | c | c | c | }
        \hline
        \textbf{Tipo} & \textbf{Tiempo de Reacción} & \textbf{Proporción}\\
        \hline
        \textbf{Rápido} & 1 segundo & 7.5\% \\
        \hline
        \textbf{Normal rápido} & 2 segundos & 25\% \\
        \hline
        \textbf{Normal} & 3 segundos & 35\% \\
        \hline
        \textbf{Normal lento} & 4 segundos & 25\% \\
        \hline
        \textbf{Lento} & 5 segundos & 7.5\% \\
        \hline
    \end{tabular}
\end{table}

\paragraph{Tipos de Vehículos}

Además de los tipos de conductores, es necesario definir los tipos de vehículos que se considerarán en el modelo. En este caso, se han definido cuatro tipos
de vehículos, que se describen en la tabla \ref{TB:VEHIC_TYPES}. Se ha considerado que los vehículos más grandes (camiones y furgonetas) tienen una longitud
mayor, y por tanto, una velocidad máxima menor que sedanes y motocicletas. Además, se ha establecido que las motocicletas son más rápidas que los sedanes,
principalmente para poder tener distintos tipos de vehículos con distintos patrones de velocidad.

\begin{table}[Tipos de vehículos]{TB:VEHIC_TYPES}{Características de los vehículos del modelo.}
    \begin{tabular}{| c | c | c | c | c |}
        % {\textwidth}{@{\extracolsep{\fill}} | c | c | c | c | c | }
        \hline
        \textbf{Tipo} & \textbf{Longitud} & \textbf{Velocidad Mínima} & \textbf{Velocidad Máxima} & \textbf{Proporción} \\
        \hline
        \textbf{Camión} & 8 metros & 55 km/h & 80 km/h & 10\% \\
        \hline
        \textbf{Furgoneta} & 4.5 metros & 65 km/h & 100 km/h & 15\% \\
        \hline
        \textbf{Sedán} & 3.5 metros & 75 km/h & 120 km/h & 55\% \\
        \hline
        \textbf{Motocicleta} & 2 metros & 90 km/h & 130 km/h & 20\% \\
        \hline
    \end{tabular}
\end{table}

\begin{table}[Tipos de conductores]{TB:DRIVER_TYPES_SPEEDS}{Características de los conductores del modelo.}
    \begin{tabular}{| c | c | c | c | c | c |}
        % {\textwidth}{@{\extracolsep{\fill}} | c | c | c | c | c | c | }
        \hline
        \textbf{Tipo} & \textbf{Factor} & \textbf{Camión} & \textbf{Furgoneta} & \textbf{Sedán} & \textbf{Motocicleta} \\
        \hline
        \textbf{Cauto} & 0.8 & 64 km/h & 80 km/h & 96 km/h & 104 km/h \\
        \hline
        \textbf{Normal} & 1.0 & 80 km/h & 100 km/h & 120 km/h & 130 km/h \\
        \hline
        \textbf{Arriesgado} & 1.1 & 88 km/h & 110 km/h & 132 km/h & 143 km/h \\
        \hline
        \textbf{Agresivo} & 1.2 & 96 km/h & 120 km/h & 144 km/h & 156 km/h \\
        \hline
        \textbf{Muy agresivo} & 1.3 & 104 km/h & 130 km/h & 156 km/h & 169 km/h \\
        \hline
    \end{tabular}
\end{table}

\paragraph{Características de la Vía}

Para poder simular el comportamiento de los conductores, es necesario definir las características de la vía en la que se desarrollarán las simulaciones.

En este caso, se ha considerado una vía simple, con dos carriles y una única dirección, sin carriles de aceleración o deceleración, en la que se permiten los adelantamientos por el carril izquierdo. La longitud
de la vía es de 5000 metros, y se ha establecido una velocidad máxima de 130 km/h, que es la velocidad máxima de los vehículos más rápidos (las motocicletas), que podrá
ser rebasada por los conductores más agresivos.

Además, se ha establecido una longitud de \textit{sección} de 500 metros, que se utilizará para estimar la carga de conductores en diferentes
segmentos de la vía y aceptar o rechazar nuevos vehículos, esto es, que si se quieren añadir nuevos vehículos a la vía, se comprobará si la carga de la sección en
la que se quiere añadir el vehículo (siempre la primera, al no contar la vía con carriles de aceleración o deceleración) es inferior a un umbral, que se detallará en
la sección \ref{SEC:EXPERIMENTALDESIGN}. En caso de que la carga sea inferior al umbral, se añadirá el vehículo a la vía, y en caso contrario, se rechazará.

Los vehículos y tiempos de reacción de estos son escogidos de forma aleatoria siguiendo los pesos probabilísticos establecidos en la columna de proporción de
las tablas \ref{TB:VEHIC_TYPES} y \ref{TB:REACTION_TYPES} respectivamente, y para cada tipo de conductor se escogerán distintas combinaciones para analizar
el comportamiento de los conductores en diferentes situaciones, detallado en la sección \ref{SEC:EXPERIMENTALDESIGN}.
La velocidad de los vehículos se establece en función del tipo de vehículo y del tipo de conductor uniformemente según la tabla \ref{TB:DRIVER_TYPES_SPEEDS}.

Por último, se ha considerado que la vía no presenta condiciones adversas, como el clima o el estado de la carretera, que podrían influir en el comportamiento de los
conductores, pero que queda fuera del alcance de este modelo.

\paragraph{Otros Parámetros}

% Funcionamiento de los accidentes

Además de los parámetros mencionados anteriormente, es necesario definir otros parámetros que permitan simular el comportamiento de los conductores en situaciones
de tráfico adversas, como accidentes o atascos.

En este caso, se ha considerado que los accidentes se producen cuando un vehículo colisiona con otro, es decir, cuando las posiciones de los vehículos quedan superpuestas
en función de la longitud de cada uno y el punto que ocupan en el espacio de una dimensión sobre la que avanzan (considerado el punto central del vehículo). En caso de que
se produzca un accidente, se detendrán los vehículos implicados en el mismo, y se establecerá un tiempo de espera de $15s$ por cada vehículo implicado en el accidente hasta
que se limpie la vía, momento en el que los vehículos implicados dejarán de formar parte de la simulación. Se calcula el tiempo de limpieza de un accidente según
la ecuación \ref{EQ:ACC_CLEANING_TIME}, donde $N_{\text{vehicles}}$ es el número de vehículos implicados en el accidente.

\begin{equation}[EQ:ACC_CLEANING_TIME]{Tiempo de limpieza de un accidente}
    t_{\text{expiration}} = 15s \cdot N_{\text{vehicles}}
\end{equation}

En ningún caso los conductores implicados reanudarán la marcha, y su velocidad se establecerá a $0 \, km/h$ hasta que se limpie la vía. De esta forma, naturalmente
surgirán atascos (además de por otras variantes como la propia saturación de la vía en condiciones de alta densidad de tráfico), que se podrán analizar en las
simulaciones. En las simulaciones, el umbral de accidentes se establecerá en $0.1$, es decir, si en algún momento la simulación supera el $10\%$ de accidentes, se
detendrá y no se considerará válida para el análisis de tiempos, aunque sí se tendrá en cuenta para el análisis de accidentes.

