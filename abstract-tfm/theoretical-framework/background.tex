El estudio del tráfico y el comportamiento de los conductores ha sido un área de interés desde hace décadas. Los modelos de tráfico,
especialmente aquellos que consideran dinámicas y comportamientos humanos, ofrecen una comprensión profunda de cómo las carreteras
y las infraestructuras se utilizan y, más importante aún, cómo pueden mejorarse para garantizar la seguridad y eficiencia.
La literatura histórica en el campo ha establecido firmemente la importancia de entender el flujo de tráfico y las condiciones
que conducen a atascos \cite{leutzbach1988, daganzo2002, sugiyama2008}.

Además, con el aumento de la urbanización y el incremento del tráfico vehicular, comprender la naturaleza y causas de los accidentes
se ha vuelto fundamental.
En general, los accidentes de tráfico representan una causa significativa de muerte no natural en el mundo, lo que refuerza la importancia de
estudiar el comportamiento del conductor \cite{lizbetin2017, siuhi2021}.