\subsubsection{Tipos de conductores}
Una dimensión crítica en el estudio del comportamiento del conductor es la tipificación basada en su agresividad o cautela. Estos tipos varían desde conductores extremadamente cautos hasta aquellos altamente agresivos \cite{james2000}. La agresividad en la conducción puede manifestarse de varias maneras, desde actos deliberados de hostilidad hasta comportamientos imprudentes que ponen en riesgo la seguridad propia y la de los demás \cite{dula2003}. La personalidad y las actitudes individuales también juegan un papel fundamental en la percepción del riesgo y el comportamiento al volante \cite{ulleberg2003}.

\subsubsection{Comportamiento agresivo y conducción arriesgada}
El comportamiento agresivo al volante a menudo se correlaciona con actitudes y comportamientos de riesgo. Estos comportamientos no solo ponen en peligro al conductor agresivo, sino también a otros usuarios de la carretera. El enojo y la agresión, por ejemplo, pueden influir considerablemente en la toma de decisiones y la percepción del riesgo al conducir \cite{deffenbacher2003}.