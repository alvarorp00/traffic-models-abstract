\subsubsection{Teoría del tráfico}
Las teorías clásicas del tráfico, como las presentadas por Leutzbach \cite{leutzbach1988} y Daganzo \cite{daganzo2002},
ofrecen un marco para entender cómo los vehículos se mueven en las carreteras y cómo se desarrollan situaciones como atascos y congestiones.

\subsubsection{Simulación de tráfico}
Para poder evaluar el impacto de diferentes conductores en el flujo de tráfico,
se han desarrollado diversos modelos de simulación \cite{fisher2011,shinar2017}.
Estos modelos permiten analizar escenarios específicos, replicando las dinámicas reales del tráfico y permitiendo la introducción de diferentes variables.
Una de las aplicaciones modernas de la simulación de tráfico es el uso de metaheurísticas, como algoritmos genéticos, para optimizar los sistemas de control de tráfico \cite{teo2010}.