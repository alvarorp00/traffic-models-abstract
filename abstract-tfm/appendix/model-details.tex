En esta sección se detallará la estructura del modelo de simulación usado en el estudio, incluyendo un diagrama de clases
y una explicación breve de las clases principales. 

\begin{figure}[Diagrama de clases del modelo]{FIG:CLASS_DIAGRAM}{
    Diagrama de clases del modelo de simulación. Se han omitido la mayoría de métodos y atributos no principales
    para facilitar la comprensión del diagrama.
}
    \image{15cm}{}{pictures/traffic-models-diagram.png}
\end{figure}

% Lista de principales clases y explicación breve de cada una

\begin{description}
    \item[Engine] \hfill \\
        Clase principal del modelo. Se encarga de inicializar el sistema y ejecutar la simulación. En cada iteración de la simulación
        se encarga de actualizar el estado del sistema y de recopilar los datos necesarios para el análisis. Contiene el modelo
        en simulación y su histórico.
    \item[Model] \hfill \\
        Clase que representa el modelo de simulación. Contiene los métodos principales para actualizar el estado del sistema.
        Es la clase que contiene los conductores, los accidentes y la vía.
    \item[RunConfig] \hfill \\
        Clase que representa la configuración de una simulación. Contiene los parámetros de la simulación, como la densidad de tráfico,
        la saturación, el número de carriles, etc. Se encarga de cargar los datos de un fichero de configuración. Se usa como parámetro
        para la clase \texttt{Engine}.
    \item[Driver] \hfill \\
        Clase que representa a un conductor. Contiene los métodos para actualizar el estado del conductor, como la velocidad, la posición,
        el tiempo de conducción, etc. Se asocia con un tipo (\texttt{DriverType}), que determina el comportamiento del conductor;
        un tiempo de reacción, \texttt{DriverReactionTime}, que determina el tiempo que tarda el conductor en reaccionar ante un accidente;
        y un tipo de vehículo, \texttt{CarType}.
    \item[DriverType] \hfill \\
        Clase que representa un tipo de conductor. Puede ser CAUTIOUS, NORMAL, RISKY, AGGRESSIVE o RECKLESS.
    \item[DriverReactionTime] \hfill \\
        Clase que representa el tiempo de reacción de un conductor. Puede ser FAST, NORMAL FAST, NORMAL, NORMAL SLOW o SLOW.
    \item[CarType] \hfill \\
        Clase que representa el tipo de vehículo de un conductor. Puede ser TRUCK, VAN, SEDAN o MOTORCYCLE.
    \item[DriverDistributions] \hfill \\
        Clase que contiene las distribuciones de probabilidad de los conductores. Se usa para generar los conductores de forma aleatoria,
        inicializar y actualizar sus velocidades, generar el tiempo que han de permanecer en el carril, etc.
    \item[Accident] \hfill \\
        Clase que representa un accidente. Contiene los métodos para actualizar el estado del accidente, como el tiempo de duración,
        el tiempo restante hasta ser limpieado y los conductores implicados.
    \item[Statistics] \hfill \\
        Clase que contiene los datos estadísticos de la simulación. Se encarga de recopilar los datos necesarios para el análisis.
\end{description}

El modelo está pensado de forma que pasando como parámetro un objeto de la clase \texttt{RunConfig}, que carga los datos de un fichero
de configuración (\textit{config.py}), se pueda modificar de forma sencilla los parámetros de este mismo y pasar este objeto como
argumento a la clase \texttt{Engine}. De esta forma se puede ejecutar el modelo con diferentes configuraciones de forma sencilla.
