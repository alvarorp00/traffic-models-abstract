\subsection{Objetivos Generales:}

\begin{enumerate}
    \item \textbf{Análisis del comportamiento de los conductores:} Profundizar en la comprensión de cómo los distintos tipos de conductores y sus características individuales (como agresividad, prudencia, experiencia, entre otros) influyen directamente en la dinámica y fluidez del tráfico vehicular.
    \item \textbf{Evaluación del rendimiento de los conductores:} Determinar cuál tipo de conductor, basado en sus características y comportamiento, no solo logra una mejor experiencia y eficiencia en la conducción a nivel individual, sino que también contribuye positivamente al flujo colectivo.
    \item \textbf{Contraste entre objetivos individuales y colectivos:} Analizar cómo ciertas conductas que pueden ser óptimas para un individuo, por ejemplo, la conducción agresiva para llegar más rápido a un destino, pueden tener repercusiones negativas en el flujo global, provocando congestiones o incrementando el riesgo de accidentes.
\end{enumerate}

\subsection{Objetivos Parciales:}

\begin{enumerate}
    \item \textbf{Tipificación de conductores y vehículos:} Antes de realizar cualquier análisis, es crucial identificar y categorizar los distintos tipos de conductores y vehículos que se integrarán en el modelo. Esto incluye aspectos como el tipo de vehículo (turismo, camión, motocicleta), características del conductor (edad, experiencia, nivel de agresividad) y otros factores que puedan influir en el comportamiento en carretera.
    \item \textbf{Desarrollo del modelo de simulación:} Crear un entorno virtual que emule fielmente las condiciones reales de las carreteras, integrando variables y parámetros relevantes que permitan la simulación de diferentes escenarios de tráfico y comportamientos de conductores.
    \item \textbf{Ejecución de simulaciones:} Una vez desarrollado el modelo, llevar a cabo múltiples simulaciones que reflejen distintos escenarios de tráfico, introduciendo variaciones en los tipos de conductores, densidades vehiculares, condiciones de la carretera y otros factores.
    \item \textbf{Análisis detallado de resultados:} Interpretar y evaluar los datos obtenidos de las simulaciones, enfocándose en determinar qué tipo de conductor o conjunto de conductores contribuye de manera más eficaz al flujo vehicular y a la seguridad vial.
    \item \textbf{Implicaciones prácticas:} Basado en los hallazgos, identificar recomendaciones y sugerencias que puedan ayudar en la formulación de políticas de tráfico y educación vial, con el objetivo de mejorar la experiencia de conducción y reducir los riesgos asociados.
\end{enumerate}