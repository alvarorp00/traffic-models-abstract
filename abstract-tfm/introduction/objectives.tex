\paragraph{Objetivos Generales}

\subparagraph{Análisis del comportamiento de los conductores}

Profundizar en la comprensión de cómo los distintos tipos de conductores y sus características (como agresividad, prudencia o experiencia) influyen
directamente en la dinámica y fluidez del tráfico vehicular.

\subparagraph{Evaluación del rendimiento de los conductores}

Determinar cuál tipo de conductor, basado en sus características y comportamiento,
no solo logra ser eficaz en la conducción a nivel individual, sino que también contribuye positivamente al flujo colectivo.

\subparagraph{Contraste entre objetivos individuales y colectivos}

Analizar cómo ciertas conductas que pueden ser óptimas para un individuo pueden tener repercusiones negativas en el flujo global,
provocando congestiones o incrementando el riesgo de accidentes.

\paragraph{Objetivos Parciales}

\subparagraph{Desarrollo del modelo de simulación}

Crear un entorno virtual que integre las variables y parámetros relevantes que permitan la simulación de
diferentes escenarios de tráfico y comportamientos de conductores. Para ello se ha de identificar y categorizar aspectos como el tipo de vehículo y
características del conductor (edad, experiencia, nivel de agresividad) que puedan influir en el comportamiento en carretera.

\subparagraph{Ejecución de simulaciones}

Una vez desarrollado el modelo, llevar a cabo múltiples simulaciones que reflejen distintos escenarios de tráfico,
introduciendo variaciones en los tipos de conductores, densidades vehiculares, condiciones de la carretera y otros factores.

\subparagraph{Análisis detallado de resultados}

Interpretar y evaluar los datos obtenidos de las simulaciones para en determinar qué
tipo de conductor o conjunto de conductores contribuye de manera más eficaz al flujo vehicular y a la seguridad vial.

\subparagraph{Implicaciones prácticas}

Basado en los hallazgos, identificar recomendaciones y sugerencias que puedan ayudar en la formulación
de políticas de tráfico y educación vial, con el objetivo de mejorar la experiencia de conducción y reducir los riesgos asociados.