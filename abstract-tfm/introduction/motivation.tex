En un mundo en que la evolución y el crecimiento urbano se encuentra en constante evolución, las vías de tráfico se han convertido
en un elemento fundamental para el desarrollo de las ciudades, y con ello de los actores implicados que las utilizan. Por ello, más allá
de la importancia de la infraestructura en sí, es necesario conocer el comportamiento de los usuarios de las vías, y cómo estos afectan
a la misma. Los accidentes de tráfico, con sus consecuencias catastróficas en pérdidas humanas, materiales, económicas y de tiempo,
se mantienen como una de las principales preocupaciones en metrópolis y carreteras de todo el mundo.

Los conductores, con sus distintas personalidades, habilidades, niveles de seguridad y experiencia, así como el propio vehículo y
las condiciones de la vía, son los principales factores que influyen en la seguridad de la misma. Por ello, es necesario conocer
cómo puede el comportamiento de estos conductores afectar a la seguridad de la vía, y cómo puede este comportamiento ser modelado
para su estudio.

El interés no es nuevo, y se han llevado a cabo múltiples estudios en el pasado que tratan de descifrar los patrones del tráfico vehicular.
Sin embargo, en un contexto donde la tecnología permite cada vez más la simulación de escenarios complejos, el análisis detallado de los comportamientos
de los conductores en un entorno controlado puede aportar detalles valiosos que permitan comprender y analizar qué escenarios son más
propensos a sufrir accidentes y cuáles obtienen mejores resultados en términos de seguridad y tiempo.