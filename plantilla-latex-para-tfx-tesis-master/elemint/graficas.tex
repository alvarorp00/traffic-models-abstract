La creación de gráficas científicas es uno de ls elementos más complejos a la hora de crear un documento científico. Utilizando el paquete de \LaTeXe\ \textbf{pgfplots} es relativamente sencillo. Sin embargo, aún así este paquete es demasiado complejo. Para simplificar su uso se han creado un conjunto de entornos y funciones con el fin de facilitar esta tarea en la mayoría de los casos.

Otro paquete de gráficas disponible es el \textbf{gnuplottex} que permite diseñar gráficas utilizando gnuplot. En este caso también se aportan algunos entornos y funciones para simplificar su uso.

En cualquier caso el autor puede acudir a estos paquetes directamente si esta clase no le aporta la funcionalidad necesaria. Este es un elemento que permanece en desarrollo y el autor debería estar atento a los cambios en futuras versiones.

\subsection{Gráficas con pgfplots}{elemint/pgfplots}

\subsection{Gráficas con gnuplot}{elemint/gnuplots}
