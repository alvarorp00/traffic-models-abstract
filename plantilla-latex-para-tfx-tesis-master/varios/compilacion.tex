Para compilar el documento se puede hacer de múltiples formas y aquí se van a contar dos posibles formas. Una compilación manual basada en pdflatex y otra basada en arara.

\section{Pdflatex}

Cuando se compila manualmente es importante saber qué hacer, en qué orden y por qué. El primer paso es compilar el fuente principal con pdflatex usando como parámetro el nombre del fuente principal (no es necesario poner la extensión .tex). Esta compilación permite guardar el en fichero del mismo nombre pero con extensión .aux información acerca de elementos sin resolver. Si no hay elementos especiales, una segunda compilación resuelve todos los elementos que habían quedado sin resolver gracias a la información almacenada en el .aux. Sin embargo, si introducimos bibliografía o acrónimos, definiciones o glosario no se realizará esa segunda compilación.

Si se usa bibliografía es necesario ejecutar el comando bibtex con el nombre del fuente principal por parámetro (ahora siempre sin extensión) que recurrirá a la información de varios ficheros auxiliares par construir la bibliografía.

De igual forma, si se usan acrónimos o definiciones es necesario ejecutar el comando glossaries con el nombre del fuente principal sin extensión que, al igual que en el caso de la bibliografía utilizará múltiples ficheros auxiliares para crear el capítulo de acrónimos o de definiciones.

Para finalizar, si se quiere crear el glosario es necesario ejecutar makeindex con el nombre del principal como parámetro y como estilo el .ist creado al compilar. El fichero con el estilo será un segundo parámetro precedido por la bandera -s.

Una vez realizados estos pasos se realizará una doble compilación con pdflatex como en el primer paso de este proceso.

Es importante acudir al .log para ver el proceso de compilación, errores y \textit{warnings} con el fin de resolver los problemas que sea necesario resolver. \LaTeXe\ es extremadamente preciso en la creación de textos y muchos \textit{warnings} no es necesario resolverlos pero sí hay que revisarlos todos.

\section{Arara}

Otra forma de compilar es usando el programa arara con el nombre del principal como paámetro. En el programa principal, cada línea de comentario que contenga tras el símbolo de comentario la palabra arara será el indicador de una acción de compilación. En el fuente principal de este documento puede verse un ejemplo que además se copia en el \cref{TB:ARARA}.

Al igual que en la compilación manual, comprobar los ficheros .log es importante.

\begin{textbox}[Ejemplo de compilación con arara.]{TB:ARARA}{Ejemplo de compilación con arara. Estos comentarios deben introducirse al principio del documento principal y permiten que arara realice las tareas indicadas en el orden que se indica.}
\% arara: clean: \{files: [tfgtfmthesisuam.aux, tfgtfmthesisuam.idx, tfgtfmthesisuam.ilg, tfgtfmthesisuam.ind, tfgtfmthesisuam.bbl, tfgtfmthesisuam.bcf, tfgtfmthesisuam.blg, tfgtfmthesisuam.run.xml, tfgtfmthesisuam.fdb\_latexmk, tfgtfmthesisuam.fls, tfgtfmthesisuam.loe, tfgtfmthesisuam.lof, tfgtfmthesisuam.lol, tfgtfmthesisuam.lot, tfgtfmthesisuam.ltb, tfgtfmthesisuam.out, tfgtfmthesisuam.toc, tfgtfmthesisuam.upa, tfgtfmthesisuam.upb, tfgtfmthesisuam.acn, tfgtfmthesisuam.acr, tfgtfmthesisuam.alg, tfgtfmthesisuam.glg, tfgtfmthesisuam.glo, tfgtfmthesisuam.gls, tfgtfmthesisuam.glsdefs, tfgtfmthesisuam.idx,  tfgtfmthesisuam.ilg, tfgtfmthesisuam.xdy, tfgtfmthesisuam.loa, tfgtfmthesisuam.gnuploterrors , tfgtfmthesisuam.mw]\} \\
\% arara: pdflatex: \{shell: yes\} \\
\% arara: makeglossaries \\
\% arara: makeindex: \{style: tfgtfmthesisuam.ist \} \\
\% arara: bibtex \\
\% arara: pdflatex: \{shell: yes\} \\
\% arara: pdflatex: \{shell: yes\} \\
\% arara: clean: \{files: [tfgtfmthesisuam.aux, tfgtfmthesisuam.idx, tfgtfmthesisuam.ilg, tfgtfmthesisuam.ind, tfgtfmthesisuam.bbl, tfgtfmthesisuam.bcf, tfgtfmthesisuam.blg, tfgtfmthesisuam.run.xml, tfgtfmthesisuam.fdb\_latexmk, tfgtfmthesisuam.fls, tfgtfmthesisuam.loe, tfgtfmthesisuam.lof, tfgtfmthesisuam.lol, tfgtfmthesisuam.lot, tfgtfmthesisuam.ltb, tfgtfmthesisuam.out, tfgtfmthesisuam.toc, tfgtfmthesisuam.upa, tfgtfmthesisuam.upb, tfgtfmthesisuam.acn, tfgtfmthesisuam.acr, tfgtfmthesisuam.alg, tfgtfmthesisuam.glg, tfgtfmthesisuam.glo, tfgtfmthesisuam.gls, tfgtfmthesisuam.glsdefs, tfgtfmthesisuam.idx,  tfgtfmthesisuam.ilg, tfgtfmthesisuam.xdy, tfgtfmthesisuam.loa, tfgtfmthesisuam.gnuploterrors , tfgtfmthesisuam.mw]\}
\end{textbox}

\section{Overleaf}

En overleaf no se puede controlar la compilación y por tanto no se puede utilizar xindy ni gnuplot. En este sentido es necesario indicar la opción overleaf y no utilizar la opción gnuplot. Si se utiliza Mendeley o Zotero puede importarse la bibliografía en un solo click si se enlazan las cuentas en la configuración de Overleaf. En el menú del proyecto de Overleaf se selecciona ``bibliografía'' y se puede importar directamente el .bib de cualquiera de estos sistemas.

\section{Permitir el uso de la shell al compilador}

Si se utiliza gnuplot para crear gráficas es necesario permitir que el compilador pueda ejecutar comandos en una shell, para ello, en el caso de pdflatex es necesario usar la bandera -shell-escape y en el caso de arara utilizar la opción de shell como puede verse en el \cref{TB:ARARA}.
