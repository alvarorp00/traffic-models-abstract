Para la creación de la bibliografía se ha elegido el uso de BibTex dado que es el que permite más versatilidad a la hora de crear bibliografía.

La bibliografía siempre aparecerá al final del documento aunque es necesario configurar su funcionamiento. Para ello se utilizará el comando \textbf{\textbackslash bibliographyconfig[estilo]\{fichero\}} que tiene un parámetro opcional que es el estilo de la bibliografía y el paraámetro obligatorio es el nombre del fichero.bib en el que está la bibliografía en formato BibTex obviando la extensión de este fichero. El estilo por defecto es el siam. Por ejemplo se debe citar con el comando \textbf{\textbackslash cite\{label\}} como en este ejemoplo de cita\cite{Narendra1990} en el que se usa la etiqueta Narendra1990 que es la etiqueta asociada al artículo en el fichero .bib correspondiente.

En la mayoría de las entradas de la base de datos se puede introducir un campo \textbf{note}. Si en ese comando se indica algo como:
\textbf{note = ``{\textbackslash}href\{file://unfichero.pdf\}\{Leer\}''}
se puede realizar el hiperenlace al documento correspondiente almacenado junto al documento creado. También se pueden realizar enlaces a páginas web\cite{Zurek:1993}.
