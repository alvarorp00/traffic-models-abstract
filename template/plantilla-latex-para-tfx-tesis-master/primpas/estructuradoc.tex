A la hora de crear un documento es muy importante organizar correctamente todos los archivos que van a contener las distintas partes del documento. En este capítulo se va a presentar cómo está organizado \LaTeXe\ y cómo debe ordanizarse correctamente un documento con este estilo.

\subsection{Organización de \LaTeXe}

Un documento de \LaTeXe\ comienza siempre por comando \textbackslash documentclass aunque previamente puede haber comentarios previos. En ese comando siempre es necesario indicar el estilo que se va a utilizar y si es necesario las opciones modificadoras del estilo. Para este estilo y poniendo algunas opciones a modo de ejemplo podría ser: \textbf{\textbackslash documentclass[epsbased,lof,loc]\{tfgtfmthesisuam\}}.

Tras este comando tenemos todos los comandos que deben presentarse en el preámbulo como se indica en los distintos capítulos y apartados.

Y para finalizar, el documento en si mismo debe estar dentro del entorno \textbf{document}, es decir, el documento empezará con un \textbf{\textbackslash begin\{document\}} y terminará con un \textbf{\textbackslash end\{document\}}.

Como indica en otros capítulos hay una gran cantidad de variables del documento que deben estar en el preámbulo, es decir, antes de declarar el principio del documento con \textbf{\textbackslash begin\{document\}}.

\subsection{Organización de la documentación}

Al declarar cada capítulo, apartado o subapartado el segundo parámetro determina el fichero que será utilizado para ese capítulo, apartado, etc. Por ello se recomienda organizar los distintos capítulos o apartados en subdirectorios de forma adecuada. En este caso el nombre del fichero que se utilizará estará precedido del \textsl{path} relativo de dicho fichero al fichero principal. También puede estar precedido de un \textsl{path} absoluto. Se puede ver a modo de ejemplo cómo está estructurado este documento.

Además hay algunas funciones importantes que permiten poner los logos, imágenes, código o datos que van a ser usados en el documento en distintos directorios.
Estas funciones son:
\begin{description}
\item [\textbackslash codesdir \{\}] Directorio donde se pondrán los códigos que se van a utilizar. Las funciones para incluir código admiten a su vez en el nombre del fichero subdirectorios de este directorio.
\item [\textbackslash logosdir \{\}] Directorio donde se pondrán los logos que se van a utilizar. Las funciones para incluir logos admiten a su vez en el nombre del fichero subdirectorios de este directorio. Este directorio no debe indicarse si se ha instalado este estilo como parte del sistema operativo.
\item [\textbackslash graphicsdir \{\}] Directorio donde se pondrán las imágenes que se van a utilizar. Las funciones para incluir imágenes admiten a su vez en el nombre del fichero subdirectorios de este directorio.
\item [\textbackslash datadir \{\} ]Directorio donde se pondrán los datos para ser graficados. Las funciones para crear gáficas \textbf{no} admiten en el nombre del fichero subdirectorios de este directorio.
\end{description}

 La función para determinar dónde están los logos no debe ser usada si este estilo ha sido instalado en el sistema operativo, sin embargo sí es necesario si este estilo es usado en un directorio de usuario y los logos se ponen en un directorio distinto del del documento principal. En cualquier caso si no se indica ninguno de estos directorios, todos estos elementos deberán estar en el mismo directorio donde está el documento principal que es el mismo donde debe realizarse la compilación.

\subsection{Otros elementos de la estructura del documento}

Cuando se compila el documento a PDF existe una serie de valores que aparecen en los metadatos de dicho fichero. Para controlar estos datos es neesario ejecutar la función \textbf{\textbackslash pdfmetavalues} que tiene cuatro parámetros obligatorios. El primero es el autor del documento, el segundo el título, el tercero el tipo de documento y el cuarto una serie de palabras clave separadas por comas. por ejemplo: \textbf{\textbackslash pdfmetavalues\{Eloy Anguiano Rey\}\{Manual de la clase LaTeX2e tfgtfmthesisuam\}\{Tesis\}\{manual, tfgtfmthesisuam, TFG, TFM, Tesis\}}.

De igual forma hay un recuadro em la portada en la que se puede poner inforación variada. Típicaente esta información será la escuela o facultad, la dirección, teléfono, email, etc. Para definir el contenido de este recuadro hay que utilizar la función \textbf{\textbackslash coverdata} que sólo tiene un parámetro, el texto a incluir en ese recuadro en el que es necesario indicar cada final de línea con un \textbackslash\textbackslash.
